\newcommand*{\yts}{y_t^\ast}
\newcommand*{\ets}{\varepsilon_t^\ast}

\section{Model}

The Stochastic Volatility (SV) model was introduced in the seminal work of~\citet{taylor1982financial}.
By choosing SV, one aims at capturing time varying and clustered volatility using an AR(1) process.
The model used in this thesis is the Stochastic Volatility with Leverage, which, additional to the AR(1) process, also models the leverage effect by letting the stock return and the increment of the log variance have a constant correlation.

\subsection{Formulation}

The SV model with leverage is, in its canonical form, as formulated in~\citet{Omori2007},
\begin{equation}
\begin{alignedat}{2}\label{form:orig_model}
y_t & = \varepsilon_t\exp\left(h_t/2\right), && \quad t=1,\dots,n, \\
h_{t+1} & = \mu+\phi(h_t-\mu)+\eta_t, && \quad t=1,\dots,n-1, \\
\begin{pmatrix}
\varepsilon_t \\
\eta_t
\end{pmatrix}
\bigg\vert\left(\rho,\sigma\right) & \sim\text{ i.i.d. }\mathcal{N}_2\left(\bm{0},\bm{\Sigma}\right), \\
\bm{\Sigma} & =
\begin{pmatrix}
1 & \rho\sigma \\
\rho\sigma & \sigma^2
\end{pmatrix},
\end{alignedat}
\end{equation}
where $n$ is the number of time points, the only observed variable is $y_t$, the demeaned log returns, and it is conditionally normally distributed, given $h_t$.
The log variance, $\bm{h}$, is the latent vector, and it constitutes an AR(1) process with mean $\mu$, persistence $\phi$ and variance $\sigma^2$.
Leverage is the fourth parameter, $\rho$, which is the correlation between $\varepsilon_t$ and $\eta_t$, i.e. the increment of the stock price and the increment of the log variance.

The first equation in~\eqref{form:orig_model} is not linear in $h_t$, which makes the model difficult to estimate. For the ease of notation, let
\begin{align*}
\yts &=\log(y^2_t), \\
d_t &=I(y_t\ge0)-I(y_t<0), \qquad\text{$y_t$'s sign,} \\
\ets &=\log(\varepsilon^2_t),
\end{align*}
thus knowing $y_t$ is equivalent to knowing the pair $(\yts, d_t)$\footnote{Except for the case $\{y_t=0\}$, which is a null set in the model, and it causes identifiability issues for $h_t$. In practice, we use $\yts =\log(y^2_t+\epsilon)$ with some small $\epsilon$ for robustness.}. By storing $d_t$ and applying $x\mapsto\log(x^2)$ to the first equation of~\eqref{form:orig_model} we get the linearised form,
\begin{align}
\begin{split}\label{form:lin_model}
\yts & = h_t+\ets, \\
h_{t+1} & = \mu+\phi(h_t-\mu)+\eta_t,
\end{split}
\end{align}
where the error term of the first equation has a $\log(\chi_1^2)$ distribution. The observed variable is $\yts$ and $h_t$ is the latent state.

\subsubsection{Other forms}

The SV model with leverage was formulated differently in~\citet{Jacquier2004}, where $\varepsilon_t$ and $\eta_{t-1}$ are correlated.
A comparison provided in~\citet{yu2005leverage} revealed that model~\ref{form:orig_model} is more attractive as it is an Euler approximation to the log-normal Ornstein--Uhlenbeck process, hence, the method that fits~\ref{form:orig_model} also fits the corresponding continuous time process with discretely sampled data.
Moreover, in the alternative specification $y_t$ is not a martingale difference sequence, and $\rho$ has two roles: leverage and the skewness of $y_t\mid y_{t-1},\dots,y_1$, which makes it more difficult to interpret its value.
Finally, an empirical comparison showed the model by~\citeauthor{Jacquier2004} to be inferior to~\ref{form:orig_model}.

\subsection{Estimation without leverage}

SV models are an attractive alternative to GARCH type models, the main difference\footnote{For a more in-depth comparison see, e.g.,~\citet{Harvey1994}.} being that while the volatility of GARCH at $t+1$ is conditionally deterministic, given the information known at $t$, it is random in SV.
On the one hand, this lets SV fit the data better in some cases~\citep{Kim1998,Chan2016}, on the other hand, it makes its estimation more difficult. In the following parts, the fitting methods for SV without leverage considered in the literature are briefly summarised.

\subsubsection{Maximum likelihood estimation}

Let $\bm{y}=(y_1,\dots,y_n)$.
In order to obtain a ML estimate for $(\phi,\sigma^2,\rho,\mu)$, we need to evaluate the likelihood function $\ell(\phi,\sigma^2,\rho,\mu\mid\bm{y})$, for which we need to integrate over the space of vector $\bm{h}$.
This is unfortunately difficult due to the non-linear dependence between $y_t$ and $h_t$, or, in the linear form, due to the non-Gaussian error term $\ets$.

The issue of non-normality was resolved in~\citet{Harvey1994} using a Gaussian approximation to $\ets$, i.e. by matching the first two moments of the $\log(\chi_1^2)$ distribution.
Then, in the resulting approximate model, using a Kalman filter to integrate over $\bm{h}$, a quasi-likelihood function can be calculated, and, after maximisation, a quasi-maximum likelihood estimate can be obtained.
This estimator is consistent and asymptotically normally distributed, but it has bad performance in small samples because the $\ets$ is poorly approximated by the normal distribution~\citep{Kim1998}.

\subsubsection{Bayesian approach}

The lack of a closed form likelihood function also means that there are no closed form posteriors for the model.
This suggests the usage of Markov chain Monte Carlo algorithms (MCMC), which, with the help of Markov chains and Bayes' theorem, make it possible to draw samples from the posterior distribution of the latent variables and the parameters.
With enough such samples we get a picture of these distributions.
For an introduction, see, e.g.,~\citet{Geyer2011} or Section~\ref{sec:estimlev}.

In~\citet{Kim1998}, two different Bayesian ideas were compared for SV without leverage based on how the latent variables are sampled.
A single move (one-at-a-time volatility update) sampler was introduced first that draws $h_t$ from $h_t\mid\bm{h}_{-t},\bm{y},\phi,\sigma^2,\mu$ one by one, where $\bm{h}_{-t}$ is $\bm{h}$ excluding $h_t$.
Due to the high intercorrelation in $\bm{h}$, slow convergence and poor mixing characterise this approach even though the algorithm used by~\citeauthor{Kim1998} performs better than the other ones in the literature~\citep{shephard1993fitting,jarquier1994bayesian,shephard1994comment,shephard1997likelihood,geweke1994bayesian}.

To avoid the issues with high intercorrelation in $\bm{h}$, a multi-move sampler was used that draws $\bm{h}$ from $\bm{h}\mid\bm{y},\phi,\sigma^2,\mu$ at once.
By approximating the marginal distribution of $\ets$ with a $K=7$ component mixture of normals,~\citeauthor{Kim1998} managed to reduce the task to the known framework of conditionally Gaussian state spaces\footnote{For an introduction see~\citet{Kitagawa1996}.}.
Since the marginal of $\ets$ does not include any model-dependent values, this mixture of normals can be specified before fitting the model.
The approximation errors to the original SV model can be corrected for by a reweighting scheme.
However, this correction did not change the results significantly due to the good choice of the mixture approximation.

Matching parameters with papers' models can go to an appendix???

\subsection{Approximate model}

Both the single move and the multi-move samplers were generalised to SV with leverage in~\citet{Omori2007}, and a particle filtering\footnote{For an introduction see~\citet{Johannes2009}.} method was also derived.
In this work, we favor MCMC methods over sequential Monte Carlo (particle filtering) due to the availability of computers with strong processors (wu cloud ???), and, due to its better sampling efficiency, the multi-move sampler was chosen over the single move sampler for fitting the model.

\subsubsection{Bivariate normal approximation}

Due to the correlation between $\varepsilon_t$ and $\eta_t$, approximating $\ets$ affects $\eta_t$ as well.
Thus,~\citeauthor{Omori2007} used a mixture of bivariate normals as an approximation to the conditional distribution of the pair $(\ets, \eta_t)$.
Let $(\xi_t,\nu_t\mid d_t,\rho,\sigma)$ denote the approximate random variable to $(\ets,\eta_t\mid d_t,\rho,\sigma)$, and let $f_X$ denote the density of the random variable $X$.

In order to derive the approximation, we first decompose the bivariate conditional density and then approximate the parts separately,
\begin{align}
f_{\ets,\eta_t\mid d_t,\rho,\sigma} &= f_{\ets\mid d_t,\rho,\sigma}\cdot f_{\eta_t\mid\ets,d_t,\rho,\sigma},\nonumber \\
&= f_{\ets}\cdot f_{\eta_t\mid\ets,d_t,\rho,\sigma}\label{eq:decomp},
\end{align}
because the marginal of $\ets$ is independent of $d_t$, $\rho$, and $\sigma$.
\citeauthor{Omori2007} now used an improved normal mixture approximation to the marginal $f(\ets)$ with $K=10$,
\begin{equation}\label{eq:ets}
f_{\xi_t}\triangleq\sum_{j=1}^{10}p_jf_{\mathcal{N}\left(m_j,v_j^2\right)},
\end{equation}
where $\mathcal{N}\left(m_j,v_j^2\right)$ denotes a normally distributed random variable with mean $m_j$ and variance $v_j^2$.
The constants $m_j$, $p_j$ and $v_j$ are specified in Table~\ref{tab:constants}.

The conditional distribution of $\eta_t$ is
\begin{equation*}
\eta_t\mid\ets,d_t,\rho,\sigma\sim\mathcal{N}\left(d_t\rho\sigma\exp(\ets/2),\sigma^2\left(1-\rho^2\right)\right),
\end{equation*}
thus, we could use
\begin{equation}\label{eq:eta}
\nu_t\mid\xi_t,d_t,\rho,\sigma\sim\mathcal{N}\left(d_t\rho\sigma\exp(\xi_t/2),\sigma^2\left(1-\rho^2\right)\right),
\end{equation}
but the term $\exp(\xi_t/2)$ introduces difficulties.
These are mitigated by a linear approximation
\begin{equation}\label{eq:etslinear}
\exp(\xi_t/2)\approx\exp(m_j/2)(a_j+b_j(\xi_t-m_j))
\end{equation}
when the $j$th mixture component is used for $\xi_t$, i.e. $\xi_t\sim\mathcal{N}(m_j,v_j^2)$.
The constants $a_j$ and $b_j$ were found by minimising the mean square norm 
\begin{equation*}
\E{\left[\exp(\xi_t/2)-\exp(m_j/2)(a+b(\xi_t-m_j))\right]^2}
\end{equation*}
w.r.t. $a$ and $b$, separately for each $j$, and they are listed in Table~\ref{tab:constants}.

\begin{table}[h!]
	\centering
	\caption{Constants of the bivariate approximation~\citep{Omori2007}.}
	\label{tab:constants}
	\begin{tabular}{cccccc}
		$j$                       & $p_j$    & $m_j$      & $v_j^2$ & $a_j$    & $b_j$    \\ \hline
		\multicolumn{1}{l|}{1}  & 0.00609 & 1.92677   & 0.11265                & 1.01418 & 0.50710 \\
		\multicolumn{1}{l|}{2}  & 0.04775 & 1.34744   & 0.17788                & 1.02248 & 0.51124 \\
		\multicolumn{1}{l|}{3}  & 0.13057 & 0.73504   & 0.26768                & 1.03403 & 0.51701 \\
		\multicolumn{1}{l|}{4}  & 0.20674 & 0.02266   & 0.40611                & 1.05207 & 0.52604 \\
		\multicolumn{1}{l|}{5}  & 0.22715 & -0.85173  & 0.62699                & 1.08153 & 0.54076 \\
		\multicolumn{1}{l|}{6}  & 0.18842 & -1.97278  & 0.98583                & 1.13114 & 0.56557 \\
		\multicolumn{1}{l|}{7}  & 0.12047 & -3.46788  & 1.57469                & 1.21754 & 0.60877 \\
		\multicolumn{1}{l|}{8}  & 0.05591 & -5.55246  & 2.54498                & 1.37454 & 0.68728 \\
		\multicolumn{1}{l|}{9}  & 0.01575 & -8.68384  & 4.16591                & 1.68327 & 0.84163 \\
		\multicolumn{1}{l|}{10} & 0.00115 & -14.65000 & 7.33342                & 2.50097 & 1.25049
	\end{tabular}
\end{table}

By combining~\eqref{eq:decomp},~\eqref{eq:ets},~\eqref{eq:eta} and~\eqref{eq:etslinear}, we get the final approximation
\begin{align*}
f_{\ets,\eta_t\mid d_t,\rho,\sigma} &\approx f_{\xi_t,\nu_t\mid d_t,\rho,\sigma}, \\
&= f_{\xi_t}\cdot f_{\nu_t\mid\xi_t,d_t,\rho,\sigma}, \\
&= \sum_{j=1}^{10}p_jf_{\mathcal{N}\left(m_j,v_j^2\right)}\cdot f_{\mathcal{N}\left(d_t\rho\sigma\exp(m_j/2)(a_j+b_j(\xi_t-m_j)),\sigma^2\left(1-\rho^2\right)\right)}.
\end{align*}

Plots about how good this approximation is ???

\subsubsection{Correcting for misspecification}\label{sec:reweight}

By using an approximate distribution for the true $f_{\ets,\eta_t\mid d_t,\rho,\sigma}$, we misspecified the model, and our draws for $\bm{h}$, $\bm{\theta}$ and $\mu$ are from an approximate posterior density as well.
One can correct this and produce draws from the true posterior $f_{\bm{h},\bm{\theta},\mu\mid\bm{y}}$ by calculating weights $w_k\ (k=1,\dots,T)$ for all the draws, and resample the sampled numbers according to the weights.
After obtaining the error terms
\begin{align*}
\xi_t^k &= y_t^\ast-h_t^k, \\
\nu_t^k &= (h_{t+1}^k-\mu^k)-\phi^k(h_t^k-\mu^k).
\end{align*}
Then, we compute the non-normalised weights
\begin{equation*}
w_k^\ast=\frac{f_{\ets,\eta_t\mid d_t,\mu,\bm\theta}\left(\xi_t^k,\nu_t^k\mid d_t,\mu^k,\bm\theta^k\right)}{f_{\xi_t,\nu_t\mid d_t,\mu,\bm\theta}\left(\xi_t^k,\nu_t^k\mid d_t,\mu^k,\bm\theta^k\right)},
\end{equation*}
Finally, we normalise the weights
\begin{equation*}
w_k=\frac{w_k^\ast}{\sum_{l=1}^{T}w_l^\ast}
\end{equation*}
to get the probabilities that we use for resampling the existing draws.
\citeauthor{Omori2007} found that the weights have quite small variance which makes the effect of this correction procedure modest.

\subsubsection[State space form]{Conditional Gaussian state space form}

Due to the normal mixture approximation, a new variable $\bm s=(s_1,\dots,s_T)$ has been introduced to the model, the vector of mixture components, one component for each time point. Given $s_t=j$, we end up with a linear model with normal errors,
\begin{equation}
\begin{alignedat}{2}\label{form:appr_model}
\yts & = h_t+\xi_t, \quad t=1,\dots,n, \\
h_{t+1} & = \mu+\phi(h_t-\mu)+\nu_t, \quad t=1,\dots,n-1, \\
\begin{pmatrix}
\xi_t \\
\nu_t
\end{pmatrix} &\overset{d}{=}
\begin{pmatrix}
m_j \\
a_j\gamma_t^j
\end{pmatrix} +
\begin{pmatrix}
v_j && 0 \\
b_jv_j\gamma_t^j && \sigma\sqrt{1-\rho^2}
\end{pmatrix}
\begin{pmatrix}
z_t^1 \\
z_t^2
\end{pmatrix}, \\
\begin{pmatrix}
z_t^1 \\
z_t^2
\end{pmatrix}
&\sim\text{ i.i.d. }\mathcal{N}_2\left(\bm{0},\bm{I_2}\right),
\end{alignedat}
\end{equation}
where $\gamma_t^j\triangleq d_t\rho\sigma\exp(m_j/2)$ and ``$\overset{d}{=}$'' means equivalence in distribution. Note that $(\xi_t,\nu_t)$ depends on $d_t,\rho,\sigma$ and $s_t$.

In order to reduce the estimation of model~\eqref{form:appr_model} to the estimation of a well-known framework, we first reformulate~\eqref{form:appr_model} equivalently as
\begin{equation}
\begin{alignedat}{2}\label{form:gauss_model}
\begin{pmatrix}
\yts \\
h_{t+1} \\
\tilde{\mu}_{t+1}
\end{pmatrix} &=
\begin{pmatrix}
h_t \\
\tilde{\mu}_t+\phi(h_t-\tilde{\mu}_t) \\
\tilde \mu_t
\end{pmatrix} +
\begin{pmatrix}
\xi_t \\
\nu_t \\
0
\end{pmatrix}, \quad t=1,\dots,T-1, \\
y_T^\ast &= h_T+\xi_T.
\end{alignedat}
\end{equation}
Since $(\xi_t,\nu_t,0)$ is a (degenerate) normal white-noise series, setting $\mu\equiv\tilde{\mu}_t$ and assuming a normal prior for $\mu$ and $h_1\mid(\mu,\sigma,\phi)$, model~\eqref{form:gauss_model} is a linear Gaussian state space (GSS) with hidden state $(h_t,\tilde{\mu}_t)$.
We copy the priors of the initial latent state used by~\citeauthor{Omori2007}, for arbitrary constant $\sigma_0$,
\begin{equation*}
\begin{pmatrix}
h_1 \\
\tilde\mu_1
\end{pmatrix} \sim
\mathcal{N}\left(
\begin{pmatrix}
\mu_0 \\
\mu_0
\end{pmatrix},
\begin{pmatrix}
\sigma^2/(1-\phi^2)+\sigma_0^2 & \sigma_0^2 \\
\sigma_0^2 & \sigma_0^2
\end{pmatrix}
\right).
\end{equation*}
Note that in~\eqref{form:gauss_model} $\tilde{\mu}_t$ is constant through $t$. \citeauthor{Omori2007} needed this ``trick'' with the inclusion of $\mu$ in the hidden state in order to match the general form of models that can be fitted via the method in~\citet{de1995simulation}.

\subsection{Estimation with leverage}\label{sec:estimlev}

\subsubsection[MCMC algorithms]{Markov Chain Monte Carlo algorithms}

The term Monte Carlo is used for the simulation of random processes, while the term Markov chain denotes sequences of random variables $X_1,X_2,\dots$ for which, for each $n\in\mathbb{N}$, the conditional distribution of $X_{n+1}$ given $X_1,\dots,X_n$ only depends on $X_n$ (it is ``memoryless'').
This way, the value of $X_n$ can be thought of as the state of the chain at time $n$, and then the distribution $f_{X_{n+1}\mid X_n}(v\mid u)$ is the probability of transition from state $u$ to state $v$ at time $n$.
Under sufficient conditions, the states of a Markov chain converge in distribution to $\pi$, called the equilibrium distribution, from every initial state having positive density~\citep{grinstead2012introduction}.

In practice, we cannot prove the conditions for the existence of $\pi$.
Instead we only have the output of the algorithm, a sequence of realisations, from which we can try to imply that we have a converged chain using, e.g., visualisations and autocorrelation functions.
\citet{Geyer2011} mentions the issues that pop up here and argues about some solution ideas.

\comment{
	These transition probabilities are called stationary if they don't depend on $n$.
	
	For the majority of the Markov chain Monte Carlo (MCMC) algorithms stationary transition probability distributions are needed.
	These are easier to handle, e.g. the joint distribution of such a Markov chain is characterised by the initial distribution of $X_1$ and on the general transition distribution from $X_n$ to $X_{n+1}$.
	
	The sequence $X_1,X_2,\dots$ itself is called stationary if for each $k\in\mathbb{N}$ and $n\in\mathbb{N}$ the joint distribution of the tuple $(X_n,\dots,X_{n+k})$ is independent of $n$.}

\subsubsection{Steps overview}

After initialising all the latent variables and parameters, there are three main steps when doing a Bayesian estimation of a conditional linear GSS.
\begin{enumerate}[start=0]
	\item Initialise $\bm h,\bm s,\bm{\theta},\mu$,
	\item Draw $\bm{s}\mid\bm{y}^\ast,\bm{d},\bm{h},\mu,\bm\theta$\label{enum:draw-s},
	\item Draw $\bm\theta,\bm h,\mu\mid\bm y^\ast,\bm d,\bm s$\label{enum:draw-other},
	\begin{enumerate}
		\item Draw $\bm\theta\mid\bm{y}^\ast,\bm{d},\bm s$\label{enum:draw-theta},
		\item Draw $\bm{h},\mu\mid\bm{y}^\ast,\bm{d},\bm\theta,\bm s$\label{enum:draw-latent},
	\end{enumerate}
	\item Goto Step~\ref{enum:draw-s}.
\end{enumerate}
Steps~\ref{enum:draw-s} and~\ref{enum:draw-other} are interchangeable in the algorithm but~\ref{enum:draw-theta} and~\ref{enum:draw-latent} are not because the former's by-products are needed for the latter.
The steps are detailed below.

If the Markov chain specified by the algorithm above has an equilibrium distribution, then it is an approximate to $\bm h,\bm s,\bm{\theta},\mu\mid\bm y$, the posterior distribution of the parameters and latent variables, and it is the true one after the reweighting scheme (Section~\ref{sec:reweight}).
So by repeating the steps sufficiently many times, after ``reaching convergence'' we obtain draws from the posterior.
With such a sample we can estimate the posterior mean or any other point estimate, calculate credible intervals or transforms of the distribution.

\subsubsection{Drawing the mixture states}

a

\subsubsection{Drawing $\rho,\sigma,\phi$}

a

\subsubsection{Drawing $\mu$ and the log variance}

a
