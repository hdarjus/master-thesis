\section{Results}

The objective of the master thesis is the comparison of the leverage effect estimated for the Chinese and for the German markets.
The SV model with leverage is fitted separately to several Chinese and German stocks and time periods.
This way, we observe differences independent of time and companies.
Furthermore, we investigate whether the leverage effect changes over time.

As it is shown in this section, our findings are mostly consistent with the literature.
In the examined period, among the chosen big corporations, the leverage effect in Germany is much stronger.
On the contrary, there are times when the return-variance correlation is even positive in China.
Also, in general, the correlation shifted to the negative direction throughout the crisis and its short term aftermath.

\subsection{Data}

A set of 10 stocks is chosen to represent China from the 50 companies currently listed on the Shanghai Stock Exchange's SSE 50 Index.
Similarly, 10 stocks are chosen to represent Germany from the current list of 30 stocks of the DAX Index from Frankfurt.
In both countries, the companies are chosen uniformly from the ones that have a price history dating back to 2004.
Tables~\ref{tab:gercompanies} and~\ref{tab:chicompanies} show the final choice.
\begin{table}[h!]
	\centering
	\begin{tabular}{lr}
		\textbf{Company name} & \textbf{Bloomberg ticker} \\
		\hline
		BASF SE & BAS GY Equity \\
		Bayerische Motoren Werke AG & BMW GY Equity \\
		Commerzbank AG & CBK GY Equity \\
		Deutsche Telekom AG & DTE GY Equity \\
		HeidelbergCement AG & HEI GY Equity \\
		Linde AG & LIN GY Equity \\
		Merck KGaA & MRK GY Equity \\
		\thead[cl]{M\"unchener R\"uckversicherungs-\\\ Gesellschaft AG in M\"unchen} & MUV2 GY Equity \\
		SAP SE & SAP GY Equity \\
		Siemens AG & SIE GY Equity
	\end{tabular}
	\caption{German companies}
	\label{tab:gercompanies}
\end{table}

\begin{table}[h!]
	\centering
	\begin{tabular}{lr}
		\textbf{Company name} & \textbf{Bloomberg ticker} \\
		\hline
		\thead[cl]{Shanghai Pudong Development\\\ Bank Co., Ltd.} & 600000 CH Equity \\
		\thead[cl]{China Minsheng Bank} & 600016 CH Equity \\
		\thead[cl]{Citic Securities Co., Ltd.} & 600030 CH Equity \\
		\thead[cl]{China United Network\\\ Communications Ltd.} & 600050 CH Equity \\
		\thead[cl]{SAIC Motor Co., Ltd.} & 600104 CH Equity \\
		\thead[cl]{China Northern Rare Earth\\\ (Group) High-Tech Co., Ltd} & 600111 CH Equity \\
		\thead[cl]{China Fortune Land\\\ Development Co., Ltd.} & 600340 CH Equity \\
		\thead[cl]{Kweichow Moutai Co., Ltd.} & 600519 CH Equity \\
		\thead[cl]{Haitong Securities Co., Ltd} & 600837 CH Equity \\
		\thead[cl]{Inner Mongolia Yili\\\ Industrial Group Co., Ltd.} & 600887 CH Equity
	\end{tabular}
	\caption{Chinese companies}
	\label{tab:chicompanies}
\end{table}

Altogether, 32 periods are used, all 3-year-long, in a moving window with step size of 3 months.
The first period is 2004/01/01-2006/12/31, the last one is 2011/10/01-2014/09/30.
This way, the dataset fully includes the Subprime Crisis of 2007/08.

\subsection{Setup}

We use the same priors in the paper at hand as~\citet{Nakajima2009}:
\begin{align*}
\frac{\phi+1}2 &\sim\text{Beta}(20,1.5), \\
\sigma^2 &\sim\text{InverseGamma}(2.25,\text{rate}=0.0625), \\
\rho &\sim\mathcal{U}(-1,1), \\
\mu &\sim\mathcal{N}(-9,1), \\
h_1\mid\phi,\sigma,\mu &\sim\mathcal{N}(\mu,\sigma^2/(1-\phi^2)).
\end{align*}
In order to achieve convergence, a burn-in of $50\,000$ draws was used, and then $100\,000$ samples were recorded for evaluation. Hence, a total of $20\times32\times(100\,000+50\,000)=96\,000\,000$ samples were drawn for data set sizes of around 750.
