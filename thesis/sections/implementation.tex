\section{Implementation}

\citeauthor{Nakajima2009} developed and published a software that was used for their calculations~\citep{nakajima2009code}.
However, that was written in the proprietary Ox language, which is only freely available with limited features for academic and teaching purposes~\citep{doornik2009object}.
Apart from investigating the behavior of leverage, this master thesis also aimed at providing a freely available solution.
This section provides details and tests about this implementation, which will be published under the GNU GPLv3 License~\citep{gplv3}.
In the meantime, the software is available upon request from the author.

\subsection{Framework}

The software is written mainly in R~\citep{rlanguage}, and some parts are in C++~\citep{iso2016iec} via the package Rcpp~\citep{rcpp2011} for the sake of efficiency.
There are two parts to the software: an R package that provides a function that runs the algorithm detailed in Section~\ref{sec:estimlev}, and a set of individual R scripts that apply that function on data, evaluate results and create plots.
Other packages used for the model are numDeriv~\citep{rnumderiv}, Matrix~\citep{rmatrix}, testthat~\citep{rtestthat}, mvtnorm~\citep{rmvtnorm} and MCMCpack~\citep{rmcmcpack}.
For data manipulation and visualisation the packages tidyr~\citep{rtidyr}, dplyr~\citep{rdplyr} and ggplot2~\citep{rggplot2} are used.

\subsection{Simulation}

Simulation tests of the model fitting package are shown in this section.

Setup

Plots