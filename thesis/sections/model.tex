\newcommand*{\yts}{y_t^\ast}
\newcommand*{\ets}{\varepsilon_t^\ast}

\section{Model}

The Stochastic Volatility model was introduced in the seminal work of~\citet{Taylor1982}.
By choosing SV, one aims at capturing time varying and seasonal volatility using an AR(1) process.
The model used in this thesis is the Stochastic Volatility with leverage, which, additional to the AR(1) process, also models the leverage effect by letting the stock return and the increment of the log variance have a constant correlation.

\subsection{Formulation}

The SV model with leverage is, as formulated in~\citet{Omori2007},
\begin{equation}
\begin{alignedat}{2}\label{form:orig_model}
y_t & = \varepsilon_t\exp\left(h_t/2\right), && \quad t=1,\dots,T, \\
h_{t+1} & = \mu+\phi(h_t-\mu)+\eta_t, && \quad t=1,\dots,T-1, \\
\begin{pmatrix}
\varepsilon_t \\
\eta_t
\end{pmatrix}
\bigg\vert\left(\rho,\sigma\right) & \sim\text{ i.i.d. }\mathcal{N}_2\left(\bm{0},\bm{\Sigma}\right), && \quad t=1,\dots,T-1, \\
\varepsilon_T &\sim\mathcal{N}(0,1), \\
\bm{\Sigma} & =
\begin{pmatrix}
1 & \rho\sigma \\
\rho\sigma & \sigma^2
\end{pmatrix},
\end{alignedat}
\end{equation}
where $T$ is the number of time points, the only observed variables are the demeaned log returns, $y_t$.
The return at time $t$ is thus conditionally normally distributed, given $h_t$.
The log variance, $\bm{h}$, is a latent vector, and it constitutes an AR(1) process with mean $\mu$, persistence $\phi$ and variance $\sigma^2$.
Leverage is the fourth parameter, $\rho$, which is the correlation between $\varepsilon_t$ and $\eta_t$, i.e.\ the increment of the stock price and the increment of the log variance.

The first equation in~\eqref{form:orig_model} is not linear in $h_t$, which makes the model difficult to estimate. For the ease of notation, let
\begin{align*}
\yts &=\log(y^2_t), \\
d_t &=I(y_t\ge0)-I(y_t<0), \qquad\text{$y_t$'s sign,} \\
\ets &=\log(\varepsilon^2_t),
\end{align*}
thus knowing $y_t$ is equivalent to knowing the pair $(\yts, d_t)$\footnote{Except for the case $\{y_t=0\}$, which is a null set in the model, and it causes identifiability issues for $h_t$. In practice, $\yts =\log(y^2_t+\epsilon)$ is used with some small $\epsilon$ for robustness.}. By storing $d_t$ and applying $x\mapsto\log(x^2)$ to the first equation of~\eqref{form:orig_model} we get the linearised form,
\begin{align}
\begin{split}\label{form:lin_model}
\yts & = h_t+\ets, \\
h_{t+1} & = \mu+\phi(h_t-\mu)+\eta_t,
\end{split}
\end{align}
where the error term of the first equation has a $\log(\chi_1^2)$ distribution. The observed variable is $\yts$ and $h_t$ is the latent state.

\subsubsection*{Other forms}

The SV model with leverage was formulated differently in~\citet{Jacquier2004}, where $\varepsilon_t$ and $\eta_{t-1}$ are correlated.
A comparison provided in~\citet{yu2005leverage} revealed that model~\eqref{form:orig_model} is more attractive as it is an Euler approximation to the log-normal Ornstein--Uhlenbeck process.
Hence, the method that fits~\eqref{form:orig_model} also fits the corresponding continuous time process with discretely sampled data.
Moreover, in the alternative specification, $y_t$ is not a martingale difference sequence, and $\rho$ has two roles: leverage and the skewness of $y_t\mid y_1,\dots,y_{t-1}$, which makes it more difficult to interpret its value.
Finally, an empirical comparison also showed the model by~\citet{Jacquier2004} to be inferior to~\eqref{form:orig_model}.
For these reasons, we use~\eqref{form:orig_model} in the paper at hand.

\subsection{Estimation without leverage}

SV models are an attractive alternative to GARCH type models, the main difference\footnote{For a more in-depth comparison see~\citet{Harvey1994}.} being that while the volatility of GARCH at $t+1$ is conditionally deterministic, given the information known at $t$, it is random in SV.
On the one hand, this lets SV fit the data better in some cases~\citep{Kim1998,kastner2016dealing,Chan2016}, on the other hand, it makes its estimation more difficult. In the following parts, the fitting methods for SV without leverage considered in the literature are briefly summarised.

\subsubsection{Maximum likelihood estimation}

Let $\bm{y}=(y_1,\dots,y_n)$.
In order to obtain an ML estimate for $(\phi,\sigma^2,\rho,\mu)$, we need to evaluate the likelihood function $\ell(\phi,\sigma^2,\rho,\mu\mid\bm{y})$, for which we need to integrate over the space of vector $\bm{h}$.
This is unfortunately difficult due to the non-linear dependence between $y_t$ and $h_t$, or, in the linear form, due to the non-Gaussian error term $\ets$.

The issue of non-normality was resolved in~\citet{Harvey1994} using a Gaussian approximation to $\ets$, i.e.\ by matching the first two moments of the $\log(\chi_1^2)$ distribution.
In the resulting approximate model, a quasi-maximum likelihood estimate can be obtained by maximising a so-called quasi-likelihood function.
That function is the result of the application of the Kalman filter that integrates over the vector $\bm{h}$.
This estimator is consistent and asymptotically normally distributed, but it has bad performance in small samples because the $\ets$ is poorly approximated by the normal distribution~\citep{Kim1998}.

\subsubsection{Bayesian approach}

The lack of a closed form likelihood function also means that there are no closed form posteriors for the model.
This suggests the usage of Markov chain Monte Carlo algorithms (MCMC), which, with the help of Markov chains and Bayes' theorem, make it possible to draw samples from the posterior distribution of the latent variables and the parameters.
With enough such samples we get a picture of these distributions.
For an introduction, see, e.g.,~\citet{Geyer2011}.

In~\citet{Kim1998}, two different Bayesian ideas were compared for SV without leverage based on how the latent variables are sampled.
A single move (one-at-a-time volatility update) sampler was introduced first. It draws $h_t$ from $h_t\mid\bm{h}_{-t},\bm{y},\phi,\sigma^2,\mu$ one by one, where $\bm{h}_{-t}$ is $\bm{h}$ excluding $h_t$.
Due to the high intercorrelation in $\bm{h}$, slow convergence and poor mixing characterise this approach even though the algorithm used by~\citet{Kim1998} performs better than the other ones in the literature~\citep{shephard1993fitting,jacquier2002bayesian,shephard1994comment,shephard1997likelihood,geweke1994bayesian}.

To avoid the issues with high intercorrelation in $\bm{h}$, a multi-move sampler was used that draws $\bm{h}$ from $\bm{h}\mid\bm{y},\phi,\sigma^2,\mu$ at once.
By approximating the marginal distribution of $\ets$ with a $K=7$ component mixture of normals,~\citet{Kim1998} managed to reduce the task to the known framework of conditionally Gaussian state spaces\footnote{For an introduction see~\citet{Kitagawa1996}.}.
Since the marginal of $\ets$ does not include any model-dependent values, this mixture of normals can be specified before fitting the model.
The approximation errors to the original SV model can be corrected for by a reweighting scheme.
However, this correction is known to have minor influence due to the good choice of the mixture approximation~\citep{Kim1998}.

\subsection{Approximate model}

Both the single move and the multi-move samplers were generalised to SV with leverage in~\citet{Omori2007}, and a particle filtering\footnote{For an introduction see~\citet{Johannes2009}.} method was also derived.
In this work, we favor MCMC methods over sequential Monte Carlo (particle filtering) due to the availability of computers with a multitude of strong processors.
Finally, due to its better sampling efficiency, the multi-move sampler was chosen over the single move sampler for fitting the model.

\subsubsection{Bivariate normal approximation}

\begin{table}[t!]
	\centering
	\begin{tabular}{cccccc}
		$j$                       & $p_j$    & $m_j$      & $v_j^2$ & $a_j$    & $b_j$    \\ \hline
		\multicolumn{1}{l|}{1}  & 0.00609 & 1.92677   & 0.11265                & 1.01418 & 0.50710 \\
		\multicolumn{1}{l|}{2}  & 0.04775 & 1.34744   & 0.17788                & 1.02248 & 0.51124 \\
		\multicolumn{1}{l|}{3}  & 0.13057 & 0.73504   & 0.26768                & 1.03403 & 0.51701 \\
		\multicolumn{1}{l|}{4}  & 0.20674 & 0.02266   & 0.40611                & 1.05207 & 0.52604 \\
		\multicolumn{1}{l|}{5}  & 0.22715 & -0.85173  & 0.62699                & 1.08153 & 0.54076 \\
		\multicolumn{1}{l|}{6}  & 0.18842 & -1.97278  & 0.98583                & 1.13114 & 0.56557 \\
		\multicolumn{1}{l|}{7}  & 0.12047 & -3.46788  & 1.57469                & 1.21754 & 0.60877 \\
		\multicolumn{1}{l|}{8}  & 0.05591 & -5.55246  & 2.54498                & 1.37454 & 0.68728 \\
		\multicolumn{1}{l|}{9}  & 0.01575 & -8.68384  & 4.16591                & 1.68327 & 0.84163 \\
		\multicolumn{1}{l|}{10} & 0.00115 & -14.65000 & 7.33342                & 2.50097 & 1.25049
	\end{tabular}
	\caption{Constants of the bivariate approximation~\citep{Omori2007}.}
	\label{tab:constants}
\end{table}

Due to the correlation between $\varepsilon_t$ and $\eta_t$, approximating $\ets$ affects $\eta_t$ as well.
Thus,~\citet{Omori2007} used a mixture of bivariate normals as an approximation to the conditional distribution of the pair $(\ets, \eta_t)$.
Let $(\xi_t,\nu_t\mid d_t,\rho,\sigma)$ denote the approximate random variable to $(\ets,\eta_t\mid d_t,\rho,\sigma)$, and let $\pi(X)$ denote the density of the random variable $X$ and $\pi(X=x)$ denote its value at $x$.

In order to derive the approximation, the bivariate conditional density is decomposed first, and then the parts are separately approximated,
\begin{align}
\pi(\ets,\eta_t\mid d_t,\rho,\sigma) &= \pi(\ets\mid d_t,\rho,\sigma) \pi(\eta_t\mid\ets,d_t,\rho,\sigma)\nonumber \\
&= \pi(\ets) \pi(\eta_t\mid\ets,d_t,\rho,\sigma)\label{eq:decomp},
\end{align}
because the marginal of $\ets$ is independent of $d_t$, $\rho$, and $\sigma$.
\citet{Omori2007} now used an improved normal mixture approximation to the marginal $\pi(\ets)$ with $K=10$,
\begin{equation}\label{eq:ets}
\pi(\xi_t)\triangleq\sum_{j=1}^{10}p_j\pi\left(\mathcal{N}\left(m_j,v_j^2\right)\right),
\end{equation}
where $\pi\left(\mathcal{N}\left(m_j,v_j^2\right)\right)$ denotes the normal density with mean $m_j$ and variance $v_j^2$.
The constants $m_j$, $p_j$, and $v_j$ were found by matching the first four moments of $\ets$ and $\exp(\ets)$ with the moments of $\xi_t$ and $\exp(\xi_t)$, and then applying a non-linear optimiser to minimise the distance of the densities~\citep{Kim1998}.
The constants are specified in Table~\ref{tab:constants}.

The conditional distribution of $\eta_t$ is
\begin{equation*}
\eta_t\mid\ets,d_t,\rho,\sigma\sim\mathcal{N}\left(d_t\rho\sigma\exp(\ets/2),\sigma^2\left(1-\rho^2\right)\right),
\end{equation*}
thus, we could use
\begin{equation}\label{eq:eta}
\nu_t\mid\xi_t,d_t,\rho,\sigma\sim\mathcal{N}\left(d_t\rho\sigma\exp(\xi_t/2),\sigma^2\left(1-\rho^2\right)\right),
\end{equation}
but the term $\exp(\xi_t/2)$ introduces difficulties.
These are mitigated by a linear approximation
\begin{equation}\label{eq:etslinear}
\exp(\xi_t/2)\approx\exp(m_j/2)(a_j+b_j(\xi_t-m_j))
\end{equation}
when the $j$th mixture component is used for $\xi_t$, i.e.\ $\mathcal{N}(m_j,v_j^2)$.
The constants $a_j$ and $b_j$ are the results of the mean square norm minimisation,
\begin{equation*}
\E{\left[\exp(\xi_t/2)-\exp(m_j/2)(a+b(\xi_t-m_j))\right]^2}
\end{equation*}
w.r.t.\ $a$ and $b$, separately for each $j$, and they are listed in Table~\ref{tab:constants}.

By combining~\eqref{eq:decomp},~\eqref{eq:ets},~\eqref{eq:eta}, and~\eqref{eq:etslinear}, the final approximation is
\begin{align*}
\pi(\ets,\eta_t\mid d_t,\rho,\sigma) &\approx \pi(\xi_t,\nu_t\mid d_t,\rho,\sigma) \\
&= \pi(\xi_t)\pi(\nu_t\mid\xi_t,d_t,\rho,\sigma) \\
&= \sum_{j=1}^{10}p_j\pi\left(\mathcal{N}\left(m_j,v_j^2\right)\right)\times \\
&\times\pi\left(\mathcal{N}\left(d_t\rho\sigma\exp(m_j/2)(a_j+b_j(\xi_t-m_j)),\sigma^2\left(1-\rho^2\right)\right)\right).
\end{align*}

\subsubsection[State space form]{Conditional Gaussian state space form}

Due to the normal mixture approximation, a new variable $\bm s=(s_1,\dots,s_T)$ is included in the model, the vector of mixture components. There is one component $s_t\in\{1,\dots,K\}$ for each time point $t$. Given $s_t=j$, linear model with normal errors is obtained,
\begin{equation}
\begin{alignedat}{2}\label{form:appr_model}
\yts & = h_t+\xi_t, && \quad t=1,\dots,T, \\
h_{t+1} & = \mu+\phi(h_t-\mu)+\nu_t, && \quad t=1,\dots,T-1, \\
\begin{pmatrix}
\xi_t \\
\nu_t
\end{pmatrix} &\overset{d}{=}
\begin{pmatrix}
m_j \\
a_j\gamma_t^j
\end{pmatrix} +
\begin{pmatrix}
v_j && 0 \\
b_jv_j\gamma_t^j && \sigma\sqrt{1-\rho^2}
\end{pmatrix}
\begin{pmatrix}
z_t^1 \\
z_t^2
\end{pmatrix}, && \quad t=1,\dots,T-1, \\
\xi_T &\overset{d}{=} m_j+v_jz_T^1, \\
\begin{pmatrix}
z_t^1 \\
z_t^2
\end{pmatrix}
&\sim\text{ i.i.d. }\mathcal{N}_2\left(\bm{0},\bm{I_2}\right), && \quad t=1,\dots,T,
\end{alignedat}
\end{equation}
where $\gamma_t^j\triangleq d_t\rho\sigma\exp(m_j/2)$ and ``$\overset{d}{=}$'' means equivalence in distribution. Note that $(\xi_t,\nu_t)$ depends on $d_t,\rho,\sigma$, and $s_t$.

In order to reduce the estimation of model~\eqref{form:appr_model} to the estimation of a well-known framework, the first two equations of~\eqref{form:appr_model} are reformulated equivalently as
\begin{equation}
\begin{alignedat}{2}\label{form:gauss_model}
\begin{pmatrix}
\yts \\
h_{t+1} \\
\tilde{\mu}_{t+1}
\end{pmatrix} &=
\begin{pmatrix}
h_t \\
\tilde{\mu}_t+\phi(h_t-\tilde{\mu}_t) \\
\tilde \mu_t
\end{pmatrix} +
\begin{pmatrix}
\xi_t \\
\nu_t \\
0
\end{pmatrix}, \quad t=1,\dots,T-1, \\
y_T^\ast &= h_T+\xi_T.
\end{alignedat}
\end{equation}
The error $(\xi_t,\nu_t,0)$ is a (degenerate) normal white-noise series. Hence, by assuming a Gaussian prior for $\left(h_1,\tilde\mu_1\right)$, model~\eqref{form:gauss_model} becomes a linear Gaussian state space (GSS) with hidden state $(h_t,\tilde{\mu}_t)$.
We copy the priors of the initial latent state used by~\citet{Omori2007}, for arbitrary constants $\mu_0$ and $\sigma_0$,
\begin{equation*}
\begin{pmatrix}
h_1 \\
\tilde\mu_1
\end{pmatrix} \sim
\mathcal{N}\left(
\begin{pmatrix}
\mu_0 \\
\mu_0
\end{pmatrix},
\begin{pmatrix}
\sigma^2/(1-\phi^2)+\sigma_0^2 & \sigma_0^2 \\
\sigma_0^2 & \sigma_0^2
\end{pmatrix}
\right).
\end{equation*}
Note that $\tilde{\mu}_t$ is constant in~\eqref{form:gauss_model} through $t$, and $\mu\equiv\tilde{\mu}_t$ is used to obtain $\mu$ from~\eqref{form:gauss_model}.
This ``trick'', the inclusion of $\mu$ in the hidden state, was the key in~\citet{Omori2007} for showing that~\eqref{form:gauss_model} is a special case of the model specified by~\citet{de1995simulation}.
The algorithm used by~\citet{Omori2007} and by the paper at hand is also heavily based on the algorithm defined by~\citet{de1995simulation}.

\subsubsection{Correcting for misspecification}\label{sec:reweight}

Let $\theta$ denote $(\sigma,\rho,\phi)$.
By using an approximate distribution for the true $\pi(\ets,\eta_t\mid d_t,\rho,\sigma)$, the model is misspecified, and the draws of $\bm{h}$, $\theta$, and $\mu$ are from an approximate posterior density as well.
This can be corrected, and draws can be produced from the true posterior $\pi(\bm{h},\theta,\mu\mid\bm{y})$.
First, the weights $w_k$ need to be calculated for all the draws $k=1,\dots,N$, and then the original sample needs to be resampled using $w_k$ as probabilities.
After obtaining the error terms
\begin{align*}
\xi_t^k &= y_t^\ast-h_t^k, \\
\nu_t^k &= (h_{t+1}^k-\mu^k)-\phi^k(h_t^k-\mu^k),
\end{align*}
the non-normalised weights are computed,
\begin{equation*}
w_k^\ast=\prod_{t=1}^{T}\frac{\pi\left(\ets=\xi_t^k,\eta_t=\nu_t^k\mid d_t,\mu=\mu^k,\theta=\theta^k\right)}{\pi\left(\xi_t=\xi_t^k,\nu_t=\nu_t^k\mid d_t,\mu=\mu^k,\theta=\theta^k\right)},
\end{equation*}
Finally, we normalise the weights
\begin{equation*}
w_k=\frac{w_k^\ast}{\sum_{l=1}^{T}w_l^\ast}
\end{equation*}
to get the probabilities that we use for resampling the existing draws.
\citet{Omori2007} found that the weights have quite small variance, which makes the effect of this correction procedure modest.
That is in line with the demonstrated high precision of the approximate distribution~\citep{Omori2007}.

\subsection{Estimation with leverage}\label{sec:estimlev}

\comment{
\subsubsection[MCMC algorithms]{Markov Chain Monte Carlo algorithms}

The term Monte Carlo is used for the simulation of random processes, while the term Markov chain denotes sequences of random variables $X_1,X_2,\dots$ for which, for each $n\in\mathbb{N}$, the conditional distribution of $X_{n+1}$ given $X_1,\dots,X_n$ only depends on $X_n$ (it is ``memoryless'').
This way, the value of $X_n$ can be thought of as the state of the chain at time $n$, and then $f_{X_{n+1}\mid X_n}(v\mid u)$ is the probability or density of transition from state $u$ to state $v$ at time $n$.
Under sufficient conditions, the states of a Markov chain converge in distribution to $\pi$, called the equilibrium distribution, from every initial state having positive density~\citep{grinstead2012introduction}.

In practice, we cannot prove the conditions for the existence of $\pi$.
Instead we only have the output of the algorithm, a sequence of realisations, from which we can try to imply that we have a converged chain using, e.g., visualisations and autocorrelation functions.
\citet{Geyer2011} mentions the issues that pop up here and argues about some solution ideas.

	These transition probabilities are called stationary if they don't depend on $n$.
	
	For the majority of the Markov chain Monte Carlo (MCMC) algorithms stationary transition probability distributions are needed.
	These are easier to handle, e.g. the joint distribution of such a Markov chain is characterised by the initial distribution of $X_1$ and on the general transition distribution from $X_n$ to $X_{n+1}$.
	
	The sequence $X_1,X_2,\dots$ itself is called stationary if for each $k\in\mathbb{N}$ and $n\in\mathbb{N}$ the joint distribution of the tuple $(X_n,\dots,X_{n+k})$ is independent of $n$.}

\subsubsection{Steps overview}

After initialising all the latent variables and parameters, there are three main steps when doing a Bayesian estimation of a conditional linear GSS.
\begin{enumerate}[start=0]
	\item Initialise $\bm h,\bm s,\theta,\mu$,
	\item Draw $\bm{s}\mid\bm{y}^\ast,\bm{d},\bm{h},\mu,\theta$\label{enum:draw-s},
	\item Draw $\theta,\bm h,\mu\mid\bm y^\ast,\bm d,\bm s$\label{enum:draw-other},
	\begin{enumerate}
		\item Draw $\theta\mid\bm{y}^\ast,\bm{d},\bm s$\label{enum:draw-theta},
		\item Draw $\bm{h},\mu\mid\bm{y}^\ast,\bm{d},\theta,\bm s$\label{enum:draw-latent},
	\end{enumerate}
	\item Goto Step~\ref{enum:draw-s}.
\end{enumerate}
Steps~\ref{enum:draw-s} and~\ref{enum:draw-other} are interchangeable in the algorithm, but steps~\ref{enum:draw-theta} and~\ref{enum:draw-latent} are not because the former's by-products are needed for the latter.
The steps are detailed below based on~\citet{Omori2007}.

If the Markov chain specified by the algorithm above has an equilibrium distribution, then it is an approximation to the posterior $\bm h,\bm s,\theta,\mu\mid\bm y$, and it is the true one after the reweighting scheme (Section~\ref{sec:reweight}).
So by repeating the steps sufficiently many times, after ``reaching convergence'', draws from the posterior are obtained.
With such a sample any point estimate, credible intervals, or transformations of the distribution can be estimated.

\subsubsection{Drawing the mixture states}

A closed form can be derived for $\pi(s_t=j\mid\bm{y},\bm{h},\mu,\sigma,\rho,\phi)$, $j=1,\dots,K$.
Let
\begin{align*}
\hat\xi_t &\triangleq y_t^\ast-h_t, \\
\hat{\nu}_t &\triangleq h_{t+1}-\mu-\phi(h_t-\mu), \\
\gamma_t^j &\triangleq d_t\rho\sigma\exp(m_j/2).
\end{align*}
Then, based on~\citet{Omori2007} with $K=10$, for $t=1,\dots,T-1$ and $j=1,\dots,10$,
\begin{equation}
\begin{aligned}[1]
\pi\left(s_t=j\mid\bm{y},\bm{h},\mu,\theta\right) &\propto P\left(s_t=j\right)\pi\left(\xi_t=\hat\xi_t\mid s_t=j\right)\times \\
&\times\pi\left(\nu_t=\hat\nu_t\mid\xi_t=\hat\xi_t,s_t=j,h_{t+1},h_t,\mu,\theta\right), \\
\pi\left(s_T=j\mid\bm{y},\bm{h},\mu,\theta\right)
&\propto P\left(s_T=j\right)\pi\left(\xi_T=\hat\xi_T\mid s_T=j\right)\times 1,
\end{aligned}
\end{equation}
where
\begin{align*}
P\left(s_t=j\right) &= p_j, \\
\pi\left(\xi_t=\hat\xi_t\mid s_t=j\right) &\propto \frac{1}{v_j}\exp\left(-\frac{\left(\hat\xi_t-m_j\right)^2}{2v_j^2}\right),
\end{align*}
and
\begin{multline*}
\pi\left(\nu_t=\hat\nu_t\mid\xi_t=\hat\xi_t,s_t=j,h_{t+1},h_t,\mu,\theta\right) \propto \\
\propto \exp\left(-\frac{\left(\hat{\nu}_t-\left(\gamma_t^j\left[a_j+b_j\left(\hat\xi_t-m_j\right)\right]\right)\right)^2}{2\sigma^2\left(1-\rho^2\right)}\right).
\end{multline*}
This way, values proportional to the posterior probabilities are obtained, so they need to be normalised.
We can use the inverse sampling method to draw $\bm s\mid\bm{y},\bm{h},\mu,\sigma,\rho,\phi$~\citep{grinstead2012introduction}.

\subsubsection{Drawing $\rho,\sigma,\phi$}

The remaining two steps are more involved and based on~\citet{de1995simulation}.
They are detailed and derived for SV with leverage in~\citet{Nakajima2009}, where SV with leverage is extended to jumps and t distributed residuals.

In this step, the Metropolis-Hastings (MH) algorithm is used to obtain a draw from $\theta\mid\bm{y}^\ast,\bm{d},\bm s$.
MH is useful when the posterior density can be evaluated up to a constant factor.
That can be done using Bayes' theorem, by factoring the posterior into the likelihood, the prior, and a constant factor: $\pi(\theta\mid\bm{y}^\ast,\bm{d},\bm s)\propto\pi(\bm{y}^\ast\mid\theta,\bm{d},\bm s)\pi(\theta)$.
The prior is usually chosen to be from a well-known family, and, based on~\cite{Nakajima2009}, the likelihood can be computed using the Kalman filter.

A proposal density $g(\theta\mid\bm{y}^\ast,\bm{d},\bm s)$ is needed for MH as well, whose support is a superset of the true posterior's support.
It is also important that the evaluation of and sampling from the proposal density are efficient.

Given everything above, an MH step in the $n$th loop first produces a candidate $\theta_\ast$ from $g(\theta\mid\bm{y}^\ast,\bm{d},\bm s)$.
Then, the new sample $\theta_{n+1}$ is chosen randomly from among the current value $\theta_n$ and the candidate $\theta_\ast$.
The so-called acceptance ratio drives the decision,
\begin{equation*}
\text{AR}=\frac{\pi(\bm{y}^\ast\mid\theta_\ast,\bm{d},\bm s)\pi(\theta_\ast)}{\pi(\bm{y}^\ast\mid\theta_n,\bm{d},\bm s)\pi(\theta_n)}\frac{g(\theta_n\mid\bm{y}^\ast,\bm{d},\bm s)}{g(\theta_\ast\mid\bm{y}^\ast,\bm{d},\bm s)}
\end{equation*}
Finally, $\theta_\ast$ is accepted with probability $\min\left\{\text{AR},1\right\}$, and $\theta_{n+1}=\theta_\ast$.
Otherwise $\theta_{n+1}=\theta_n$ remains the state.

\citet{Nakajima2009} chose the proposal to be $\mathcal{N}(c, C)$ truncated to the region $R=\left\{(\sigma,\rho,\phi)\mid\sigma>0,-1<\rho<1,-1<\phi<1\right\}$.
The mean vector $c$ and the covariance matrix $C$ are calculated in each loop such that the proposal's log density is the second order Taylor expansion of the true log density around the true log density's mode.\footnote{
	Even though the true posterior density can only be evaluated up to a (positive) constant factor, the Taylor approximation can be still calculated.
	The reason is that the mode is invariant under positive scaling, and differentiation is only applied to the log density, i.e.\ constant multipliers vanish.}
More precisely,
\begin{align*}
c &= \hat{\theta}+Cv, \\
C^{-1} &= -\frac{\partial^2\log\pi(\theta\mid\bm{y}^\ast,\bm{d},\bm s)}{\partial\theta\partial\theta^\prime},
\end{align*}
where
\begin{align*}
v &= \frac{\partial\log\pi(\theta\mid\bm{y}^\ast,\bm{d},\bm s)}{\partial\theta}, \\
\hat{\theta} &= \text{arg}\max_\theta\pi(\theta\mid\bm{y}^\ast,\bm{d},\bm s).
\end{align*}

\subsubsection{Drawing $\mu$ and the log variance}

Both $\mu$ and $\bm h$ are sampled using by-products of the Kalman filter used when calculating $\pi(\bm{y}^\ast\mid\theta_{n+1},\bm{d},\bm s)$ for the AR.
The new value of $\mu$ comes by simply drawing from $\mathcal{N}(q_{T+1},Q_{T+1})$, where $q_{T+1}$ and $Q_{T+1}$ are readily available at this point.
In order to get the latent vector $\bm h$, we first sample $\bm{\nu}\mid\bm{y}^\ast,\bm{d},\theta,\bm s$ using a Gaussian simulation smoother~\citep{fruhwirth1994data,carter1994gibbs} and then reconstruct $\bm h$ from the formulas in~\citet{de1995simulation}.

%%%%%%%%%%%%%%%%%%%%%%%%%%%%%%%%%%%%%%%%%%%

\comment{
	\begin{align}
	\pi(s_t=j\mid\bm{y},\bm{h},\mu,\theta)
	&= \pi(s_t=j\mid\bm{y^*},\bm{d},\bm{h},\mu,\theta),\nonumber \\
	&\overset{(1)}{=} \pi(s_t=j\mid y^*_t,d_t,h_{t+1},h_t,\mu,\theta),\nonumber \\
	&\overset{(2)}{=} \pi(s_t=j\mid \xi_t,d_t,\nu_t,\mu,\theta),\nonumber \\
	&\propto P(s_t=j)\cdot\pi(\xi_t,\nu_t\mid s_t=j,d_t,\mu,\theta),\nonumber \\
	&\overset{(3)}{=} P(s_t=j)\cdot\pi(\xi_t=y^*_t-h_t\mid s_t=j)\cdot\nonumber \\
	&\qquad\cdot\pi(\nu_t=h_{t+1}-\mu-\phi(h_t-\mu)\mid \xi_t,s_t=j,d_t,\mu,\theta),\nonumber \\
	&\propto p_j\frac{1}{v_j}\exp\left(\frac{(y^*_t-h_t-m_j)^2}{2v_j^2}\right)\cdot \\
	&\qquad\exp\left(\dots\right)
	\end{align}
}
