\section{Introduction}

Modelling the variance\footnote{In this document variance means variance of the returns and volatility is the standard deviation of returns.} of the rate of return on stocks is a cornerstone of modern finance.
Variance is a basic building block of risk measurement and prediction uncertainty, and capturing empirical facts about its behaviour is the aim of many theorists~\citep{Christie1982}.
This master thesis is concerned with the so-called leverage effect, also called asymmetric volatility, that captures the correlation between returns and changes in volatility.

\subsection{The anti-leverage effect in China}

The leverage effect is, by definition, the negative relationship between the change in price and the change in volatility.
It was first described by~\citet{black1976studies}, whose explanation was the connection to firms' leverage ratio (LR), i.e.\ the market equity to debt ratio.
Asymmetric volatility can be deduced in basic structural models of corporate finance, and the effect has also been shown using both realised volatility and implied volatility estimates of various models~\citep{Christie1982,french1987expected,Harvey1996,Bouchaud2001}.

The Chinese financial market has been traditionally different from western systems mainly due to the differently regulated environment~\citep{GORDON2003}.
That might also lead to counter-intuitive consequences that are not in line with widely used assumptions.
\citet{Shen2009} compared the leverage effect in Germany and in China, and they found a positive correlation between returns and volatility in the latter, contradicting the previously mentioned works resulting in negative correlation.


\subsection{Research objective}

The main question addressed by this master thesis:
\begin{center}
	Can the anti-leverage effect be observed in the Chinese market?
\end{center}
The question is a reflection to~\citet{Shen2009}, it is answered after choosing an appropriate model that estimates the leverage effect.
Inspired by~\citet{Christensen2015}, time-variability of the effect is also examined around the Global Financial Crisis of 2007.
