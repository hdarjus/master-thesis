\section{Introduction}

Modelling the variance\footnote{In this document variance means variance of the returns and volatility is the standard deviation of returns.} of the rate of return on stocks is a cornerstone of modern finance.
Variance is a basic building block of risk measurement and prediction uncertainty, and capturing empirical facts about its behaviour is the aim of many theorists~\citep{Christie1982}.
A well-known fact is the seasonality of volatility~\citep{schwert1989why}.
This master thesis is concerned with another empirical fact, the so-called leverage effect, also called asymmetric volatility, that captures the correlation between volatility and return.

\subsection{The anti-leverage effect in China}

The leverage effect is, by definition, the negative relationship between the change in return and the change in volatility.
It was first described by~\citet{black1976studies}, whose explanation was the connection to firms' leverage ratio, i.e. the market equity to debt ratio.
Asymmetric volatility can be deduced in basic structural models of corporate finance~\citep{Christie1982}.
The effect has also been shown using both realised volatility and implied volatility estimates of various models~\citep{Bouchaud2001,Harvey1996,Christie1982,french1987expected}.

The Chinese financial market has been traditionally different from western systems mainly due to the differently regulated environment~\citep{GORDON2003}.
That also leads to counter-intuitive consequences that are not in line with widely used assumptions.
\citet{Shen2009} compared the leverage effect in Germany and in China, and they found a positive correlation between returns and volatility, contradicting the previously mentioned works resulting in negative correlation.


\subsection{Objective}

The main question addressed by this master thesis:
\begin{center}
	Can the Chinese anti-leverage effect be confirmed in different setups?
\end{center}
The question is a reflection to~\citet{Shen2009}, it is answered after choosing an appropriate model that estimates the leverage effect.
As a by-product, a result is acquired about the leverage effect changing consistently in time across firms.
To the best of my knowledge, that question has not been investigated yet in the literature. ???
